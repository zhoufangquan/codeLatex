\documentclass[]{article}
\usepackage[UTF8]{ctex}	% 中文语言包
\usepackage{setspace}
\usepackage{geometry}
\geometry{a4paper,left=2cm,right=2cm,top=1cm,bottom=1.5cm}
\usepackage{fontspec}
\setmainfont{Comic Sans MS}  % 正文英文字体的设置

%opening
\title{英语课文翻译和解释-U6}
\author{周方全}

\begin{document}

\maketitle

\section{A 原文}
\begin{spacing}{1.75}
{\Large
	\subsection{}
	Within the first few pages of Amy Chua's Battle Hymn of the Tiger Mother; it is easy to see why her parenting memoir caused such a media frenzy in the days following its publication. "The Chinese mother believes that schoolwork always comes first; an A-minus is a bad grade; your children must be two years ahead of their classmates in math," Ms. Chua writes. She continues: "the only activities your children should be permitted to do are those in which they can eventually win a medal; and that medal must be gold."
	\subsection{}
	A second-generation Chinese immigrant, Ms. Chua was born in Illinois and married an American. When their children were born, the couple agreed they would be brought up according to their father's religion-Judaism-and their mother's parenting model-Chinese. Her tough-love prescription taps into parental insecurities about the way children are raised and schooled. But the book's popularity is also down to the fact that it offers an insight into why Chinese children seem to excel in school.
	\subsection{}
	This is not just a US phenomenon. In the UK, British-Chinese children significantly outperform their peers. Chinese children make up only 0.4 per cent of the secondary school intake, but more than 25 per cent of them are on the gifted and talent- ed register. This compares with 15 per cent of white children and 15.9 per cent of mixed-race children. This could be put down to relative levels of affluence, were it not for the fact that the achievement gap between rich and poor among British-Chinese children is smaller than in any other ethnic group. A study of GCSE results between 2005 and 2007 by academics at London University's Institute of Education and Queen Mary found that only 5 percent fewer British-Chinese children on free school meals got five A$^+$-C grades at GCSE than those not entitled to free school meals. Among white British pupils, that gap is 32 per cent.
	\subsection{}
	So why do British-Chinese pupils not only do so well, but also seem so immune to the socio-economic factors that are the bane of teachers' lives up and down the country? For Ms. Chua, the answer is that their parents expect nothing less and their children thrive under the pressure.
	\subsection{}
	This may seem an extreme parenting model, but it also rings true in the experience of many teachers. Ian Warwick, a teacher for 20 years and now the director of London Gifted and Talented, says the parents of his Chinese pupils are extremely driven and above all, want their children to succeed. "British- Chinese success is an awful lot to do with cultural background and parental expectation," he says. "I have worked in schools which had 130 languages spoken. If you had a Chinese kid, you knew they were going to do well." He struggles to think of a single Chinese child who was not a high achiever.
	\subsection{}
	Until last year, Jane McGowan, a secondary languages teacher, worked at one of Northern Ireland's top-performing grammar schools. Her experience of Chinese pupils also largely matched Ms. Chua's model. "We had quite a high proportion of Chinese pupils and in general they would be really disciplined," she says. But although this attribute was widespread, she says it was not universal. Mrs. McGowan, who taught in Belfast, says not every Chinese pupil was a high achiever. "The ones who weren't had in some way rebelled," she says.
	\subsection{}
	Becky Francis and Louise Archer from the Institute for Policy Studies in Education at London Metropolitan University have extensively researched ethnic-minority achievement, interviewing pupils, teachers and parents. Although Chinese parents are typically very interested in their children's education, Dr. Francis says their approach is quite different from that of the "pushy parent" familiar to many UK teachers. British parents who want their children to succeed are more likely to blame the school or teacher rather than their child and will happily come knocking on the teacher's door to discuss the ins and outs of their little darling's education. In contrast, she says, teachers may not get to know the parents of their Chinese pupils very well and are unlikely to learn much about their home life. Part of this is a language barrier parents cannot be involved in the school if they cannot communicate well with the teachers. "But they were often engaged in other ways," says Dr. Francis.
	\subsection{}
	As well as their high expectations, Chinese parents see their financial contribution as crucial, giving them the right to ask so much from their child. "Many of the Chinese parents interviewed, even those of quite impoverished families, work all the hours in the takeaway or the family business for extra tuition for their kids," Dr. Francis explains. "There is a hell of a lot of work going on towards their kids' achievement, but not necessarily in the expected ways that would be familiar to a white middle-class family."
	\subsection{}
	Many of the prerequisites for the Tiger Mother approach appear to be in place already in the British education system. The last government launched the Every Child Matters initiative in order to promote personalised learning,  the ethos of which is still in place in schools today. High-achieving children are registered as gifted and talented and receive additional opportunities and coaching to help them on their way.
	\subsection{}
	But contrary to the intensive regime extolled by Ms. Chua, the British approach is more about nurturing and looking after children' s emotional needs, says Denise Yates, chief executive of the National Association for Gifted Children (NAGC). The difference between helping children fulfil their potential and a relentless drive for higher grades is a subtle one, but it is there nonetheless. "The most important thing is for these children to have confidence, and the A$^+$ should come if they are thriving," Ms. Yates says.  Grades are important to teachers and parents, but above all, pupils should not feel pressured to get the best grades all the time." More and more parents believe that as long as their child is happy and has friends, the achievement will come next."
	\subsection{}
	The theory that children do not benefit from too much pressure and that bright youngsters suffer under the weight of their talents is supported by a NAGC survey of gifted and talented children which found that many of them push themselves to do well. "What we are finding is that the biggest mental health problem is linked to a fear of failure and perfectionism," says Ms. Yates. "There is a high proportion of bright children, mainly girls, who suffer from anorexia and who feel quite isolated. Quite simply, they are not on the same wavelength as children in their class and they may prefer to mix with older children." Ms. Chua's model provides little room for concern about children's emotional health. "Western parents are concerned about their children's psyches. Chinese parents aren't," she writes. "They assume strength, not fragility, and as a result behave very differently."
	\subsection{}
	Anxiety over eating habits provides a case in point. Whereas some parents would regard how much their child weighs-whether it is too much or too little—as a sensitive subject, Ms. Chua says Chinese parents have no such qualms. "Chinese mothers can say to their daughters: 'Hey fatty—lose some weight'," she says. "By contrast, Western parents have to tiptoe around the issue, talking in terms of 'health' and never mentioning the f- word, and their kids still end up in therapy for eating disorders and negative self-image."		
	
	}
\end{spacing}

\section{A 翻译}
\begin{spacing}{1.75}
{\Large 
	\subsection{}
	在蔡美儿的虎妈战歌;的前几页中,就不难看出为什么她的育儿回忆录在出版后的几天里会引起媒体的如此狂热。蔡美儿写道:"中国母亲认为功课总是排在第一位的,A-是一个糟糕的成绩,你的孩子的数学必须比他们的同学早两年。"(;;)。她继续说:"你的孩子唯一应该被允许做的活动是那些他们最终可以赢得奖牌;的活动,而那枚奖牌必须是金牌。"
	\subsection{}
	蔡美儿是第二代华裔移民,出生在伊利诺伊州,嫁给了一名美国人。当他们的孩子出生时,这对夫妇同意按照父亲的宗教犹太教和母亲的中国育儿模式来抚养他们。她的严厉爱药方利用了父母对孩子的养育和教育方式的不安全感。但这本书的受欢迎程度也归因于这样一个事实,即它提供了一个洞察为什么中国儿童似乎在学校表现优异的事实。
	\subsection{}
	这不仅仅是美国的现象。在英国,英裔中国儿童的表现远远超过他们的同龄人。中国儿童只占中学入学人数的0.4\%,但其中超过25\%的人在资优和人才培养名册上。相比之下,白人儿童的比例为15\%,混血儿童的比例为15.9\%。这可以归因于相对富裕的水平,如果不是因为英裔中国儿童之间的贫富差距比其他任何种族的孩子都要小。伦敦大学教育学院和玛丽女王大学的学者对2005年至2007年期间的GCSE成绩进行了研究,结果发现,在免费学校用餐的英裔华裔儿童在GCSE获得5个A+C成绩的人数只比那些没有资格享受免费学校用餐的孩子少5\%。在英国白人学生中,这一差距为32\%。
	\subsection{}
	那么,为什么英裔华裔学生不仅表现如此出色,而且似乎对社会经济因素如此免疫,而这些社会经济因素是困扰中国各地教师生活的祸根?对蔡美儿来说,答案是父母的期望不会降低,孩子在压力下茁壮成长。
	\subsection{}
	这可能看起来是一种极端的育儿模式,但在许多教师的经历中也是如此。当了20年教师的伊恩·沃里克(Ian Warwick)表示,他的中国学生的父母非常有干劲,最重要的是希望他们的孩子取得成功。他说:"英中关系的成功与文化背景和父母的期望有很大关系。我曾在使用130种语言的学校工作过。"如果你有一个中国孩子,你就知道他们会做得很好。"他很难想象有哪一个中国孩子不是很有成就。
	\subsection{}
	直到去年,简·麦高恩(Jane McGowan)还是一名中学语言教师,在北爱尔兰一所表现最好的文法学校工作。她在中国学生身上的经历也与蔡美儿的模式大体相符。她说:"我们有相当高比例的中国学生,总的来说,他们会很守纪律。"但她说,尽管这一属性很普遍,但并不是普遍存在的。在贝尔法斯特任教的麦高恩女士说,并不是每个中国学生都取得了很好的成绩。她说:"那些没有学过的学生在某种程度上是叛逆的。"
	\subsection{}
	伦敦大都会大学教育政策研究所的贝基·弗朗西斯和路易丝·阿彻对少数民族的成就进行了广泛的研究,采访了学生、教师和家长。弗朗西斯博士表示,尽管中国家长通常对孩子的教育非常感兴趣,但他们的做法与许多英国教师熟悉的"咄咄逼人的家长"截然不同。希望孩子成功的英国父母更容易责怪学校或老师,而不是他们的孩子,他们会很高兴地敲开老师的门,讨论他们心爱的小宝宝教育的来龙去脉。相比之下,她说,教师可能不太了解中国学生的父母,也不太可能对他们的家庭生活有太多了解。部分原因在于,如果家长不能与老师很好地沟通,他们就不能参与学校的工作。弗朗西斯博士说:"但他们经常以其他方式参与进来。"
	\subsection{}
	除了寄予很高的期望外,中国父母也认为自己的经济贡献至关重要,这让他们有权向孩子提出如此多的要求。"许多受访的中国父母,甚至包括那些来自相当贫困家庭的父母,为了给孩子交额外的学费,整天都在做外卖或做家族生意,"弗朗西斯博士解释道。"为了孩子的成就,有大量的工作在地狱里进行,但不一定是以白人中产阶级家庭所熟悉的预期方式。"
	\subsection{}
	虎妈教学法的许多先决条件似乎已经在英国的教育系统中到位了。上届政府发起了"每个孩子都很重要"倡议,以促进个性化学习,这一精神至今仍在学校中流传。成绩优异的孩子被登记为有天赋和才华的孩子,并获得额外的机会和指导,帮助他们踏上前进的道路。
	\subsection{}
	但全美天才儿童协会 (National Association for Gited Children, NAGC) 首席执行官丹尼斯·耶茨(Denise Yates)表示,与蔡美儿所推崇的密集制度相反,英国的做法更多的是培养和照顾儿童的情感需求。帮助孩子发挥潜能和坚持不懈地追求更高分数之间的区别是微妙的,但它仍然存在。耶茨说:"对这些孩子来说,最重要的是要有信心,如果他们茁壮成长,就应该拿到A$^+$。"对老师和家长来说,成绩很重要,但最重要的是,学生不应该感到总是要拿到最好的成绩的压力。"越来越多的家长相信,只要孩子快乐,有朋友,成绩就会排在第二位。"
	\subsection{}
	儿童不会从太大的压力中受益,聪明的孩子在天赋的重压下受苦,这一理论得到了NAGC对天才儿童的调查的支持,该调查发现,他们中的许多人都会强迫自己做好。耶茨说:"我们发现,最大的心理健康问题与对失败和完美主义的恐惧有关。聪明的儿童(主要是女孩)中有很高比例患有厌食症,并感到非常孤立无援。"他说,"我们发现,最大的心理健康问题与害怕失败和完美主义有关。聪明的儿童,主要是女孩,患有厌食症,感到非常孤立。"这一理论得到了NAGC对天才儿童的调查的支持。很简单,他们和班上的孩子不一样,他们可能更喜欢与年龄较大的孩子交往。"蔡美儿的模式没有提供多少空间来关心孩子的情绪健康。"西方父母担心孩子的心理。"中国父母不是,"她写道,"他们认为自己是坚强的,而不是脆弱的,因此他们的行为非常不同。"
	\subsection{}
	对饮食习惯的焦虑就是一个很好的例子。蔡美儿说,一些父母会把孩子的体重--无论是太多还是太少--视为一个敏感话题,但中国父母没有这样的顾虑。"中国母亲可以对女儿说:'嘿,胖子--减肥吧',"她说。"相比之下,西方父母不得不小心翼翼地绕过这个问题,谈论'健康',从来不提f-word这个词,他们的孩子最终仍在接受饮食失调和负面自我形象的治疗。"
	
}
\end{spacing}

\end{document}

