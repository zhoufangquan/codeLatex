\documentclass[12pt]{article}
\usepackage[UTF8]{ctex}	% 中文语言包
\usepackage{setspace}
\usepackage{geometry}
\geometry{a4paper,left=2cm,right=2cm,top=1cm,bottom=1.5cm}
\usepackage{fontspec}
\setmainfont{Times New Roman}  % 正文英文字体的设置

%opening
\title{英语课文翻译和解释-U6}
\author{周方全}

\begin{document}
	
	\maketitle
	
	\section{A 原文}
	\begin{spacing}{1.5}
		{\large
			\subsection{}
			
			
			\subsection{}
			\textbf{Individuality vs. Interdependence}
			
			Cultures differ in how much they encourage individuality and uniqueness vs. conformity and interdependence. Individualistic cultures stress self-reliance, decision-making based on individual needs, and the right to a private life. In collectivist cultures absolute loyalty is expected to one's immediate and extended family/tribe. The term familism is often used to describe the dominant social pattern where decision-making processes emphasize the needs of the family/group first, and the concept of having a "private life" may not even exist.
						
			\subsection{}
			\textbf{Nuclear vs. Extended Family Models}
			
			3 In western cultures, and particularly in European American culture, families typically follow a nuclear model comprised of parents and their children. When important healthcare-related decisions must be made, it is usually the parents who decide, though children are raised to think for themselves and are encouraged to act as age-appropriate decision makers as well. Upon reaching adulthood, when parental consent is no longer an issue, young American adults may choose to exercise their right to privacy in health care matters. This is markedly different from collectivist cultures that adhere to an extended family model. In cultures such as American Indian, Asian, Hispanic, African, and Middle Eastern, individuals rely heavily on an extended network of reciprocal relationships with parents, siblings, grandparents, aunts and uncles, cousins, and many others. Many of these people are involved in important health care decisions, including some who are unrelated to the patient through blood or marriage. For example, in some Hispanic families the godparents play a critical role. In American Indian families, tribal leaders, the elderly, and medicine men/women are key individuals to be consulted before important decisions are made.
			
			\subsection{}
			\textbf{Multi-generational Households}
			
			It is very common for families in collectivist cultures to establish multi-generational households. (This is less true when a family becomes acculturated in the United States or other western countries where privacy is more highly valued and in cases where socio-economic gains create opportunities for greater independence.) In most multi-generational households, there are at least three generations living together; the grandparents are expected to live under the same roof as their adult children and grandchildren. In multi-generational households the family of orientation (one's self, siblings, and parents) often takes precedence over the family of procreation (one's self, spouse, and offspring). This is the reverse of how European American family households usually function. In traditional Asian families, it is the oldest male in the family who brings his bride to live with his parents. The daughter-in-law is often expected to be submissive to her mother-in-law who rules the roost. In Hispanic families, grandparents from either side may live under that same roof as their children and grandchildren. Mothers often gain a great deal of support from the grandmothers in domestic matters, but this varies depending on the dynamics unique to each family.
			
			\subsection{}
			It is extremely important for health care providers to ask who lives in a patient's household in order to better understand how relationships are structured. Who are the authority figures? In Asian and Hispanic traditional families, the father is the main authority figure. He will most often make decisions about matters outside the home, speaking for the family in public settings and signing consent forms. It is usually a female figure who takes charge of domestic life. Depending on the family, this matriarch may be the mother, but it may be the mother's mother. Thus healthcare providers need to ask the mother, "Who gives you advice about raising your children?” And “who will participate in making important decisions?” In Asian and Hispanic families especially, grandmothers often decide about using traditional medicines and healing practices, thus having enormous influence on patient compliance.
			
			\subsection{}
			\textbf{Role Flexibility \& Kinship}
			
			In dealing with culturally diverse families it is useful for health care professionals to understand the basic concepts of role flexibility and kinship and how these affect family dynamics. American kinship structure is bilateral; we are not " more related" to our father's family than our mother's, or vice versa. In unilineal cultures, family membership is traced either through a male or female ancestor. In the Middle East, for example, a patrilineal pattern is established so family belonging is passed via the father's side. Some American Indian cultures, like the Navaho and Hopi-tribes', are matrilineal cultures, passing membership through the mother's family. In the Navaho tribe, property and privilege are passed from male to male, but it is the mother's brother who will pass both to his own sister's children. Thus it makes sense that a Navaho maternal uncle might bring his nephew into the hospital expecting to be empowered to sign an informed consent.
			
			\subsection{}
			Similarly, in both American Indian and African American families, role flexibility can be an important issue. It is not uncommon for Native American grandparents to raise grandchildren while the parents leave the reservation to find work. In African American families, the mother sometimes plays the role of the father and thus functions as the head of the family. In addition, older children sometimes function as parents or caretakers for younger children. The concept of role flexibility among African American families can be extended to include the parental role assumed by grandfather, grandmother, aunts, and cousins. (Boyd-Franklin, 1989) It is a good idea to determine if older children will be involved in patient care and to include them when possible in patient care training. This is important to consider for all multi-generation households.
			
			\subsection{}
			\textbf{Family Dynamics and Acculturation}
			
			Finally, it is important to consider the enormous stresses families encounter in the process of acculturation due to sudden and radical shifts in family dynamics. Parents in a recently migrated family often are aligned with the culture of the country of origin, while their offspring are likely to adapt to the dominant culture more rapidly. This often leads to inter-generational conflicts. For example, a father may lose his traditional role as the head of the family if his wife begins to work outside the home, earning income and greater independence. Similarly, if his children quickly adopt the attitudes and values of the new dominant culture, he may find it harder to communicate with them. Both parents and grandparents may feel a loss of status due to language barriers, especially if their children learn the language of the dominant culture more quickly. This can be especially problematic in healthcare settings where responsibility is shifted to younger family members who can navigate the health care system better than their parents can. In cases where children are able to communicate with health care workers in English, they may be asked to interpret for their parents. This leads to a host of potential problems for the family, including feelings of shame and betrayal that children would relay information of a personal nature to someone outside the family. This is one of the main reasons children should not be used as interpreters.
			
			\subsection{}
			\textbf{Summary}
			
			Because cultures adapt and change, making assumptions about family dynamics is problematic; families in the United States today from all cultures display a variety of configurations. Arguably, there is no longer any such thing as a "typical" family. One can, however, expect that families from more traditional cultures not acculturated in U.S. ways will tend to value familism and display family structures that are quite different from the middle-class European American family model. There are many aspects of culturally-based family dynamics not addressed within the scope of this article. Some of the best resources for learning more about cross-cultural family dynamics come from the mental health and child development fields.
								
		}
	\end{spacing}
	
	\section{A 翻译}
	\begin{spacing}{1.25}
		{\large 
			\subsection{}
			
			\subsection{}
			\textbf{个性与相互依赖}
			
			文化的差异在于它们鼓励个性和独特性,而不是一致性和相互依赖性。个人主义文化强调自力更生,基于个人需要的决策,以及享有私人生活的权利。在集体主义文化中,对直系亲属或部落的绝对忠诚是必须的。”家庭主义”一词经常用来描述占主导地位的社会模式,其中决策过程首先强调家庭/群体的需要,而拥有”私人生活”的概念甚至可能根本不存在。
			\subsection{}
			\textbf{核心家庭模型和扩展家庭模型}
			
			在西方文化中,尤其是在欧美文化中,家庭通常遵循由父母和孩子组成的核心模式。当必须做出与医疗保健相关的重要决定时,通常是由父母决定,尽管孩子从小就被教育要独立思考,并被鼓励成为与年龄相适应的决策者。成年后,当父母同意不再是问题时,年轻的美国成年人可以选择在医疗事务上,行使他们的隐私权。这与坚持大家庭模式的集体主义文化明显不同。在美洲印第安人、亚洲人、西班牙人、非洲人和中东人等文化中,个体严重依赖于与父母、兄弟姐妹、祖父母、姑姑、叔叔、堂兄弟姐妹和许多其他人建立的互惠关系网络。其中许多人都参与了重要的医疗保健决策,包括一些因血缘或婚姻关系与病人无关的人。例如,在一些西班牙家庭中,教父教母扮演了关键角色。在美国印第安人的家庭中,部落首领,老人,以及男女巫医,都是在做出重要决定前,需要咨询的关键人物。
			\subsection{}
			\textbf{多代同堂的家庭}
			
			在集体主义文化中,家庭建立多代同堂的家庭非常普遍。(如果一个家庭在美国或其他重视隐私的西方国家受到文化的影响,而且社会经济利益创造了更大的独立机会,这种情况就不那么正确了。)在大多数多代同堂的家庭中,至少有三代人生活在一起; 祖父母和成年子女及孙子女应该生活在同一屋檐下。在多代同堂的家庭中,有取向的家庭(自己、兄弟姐妹和父母)通常优先于有生育能力的家庭(自己、配偶和子女)。这和欧美家庭通常的运作方式正好相反。在传统的亚洲家庭里,是家里最年长的男人,带着他的新娘和他的父母住在一起。媳妇通常被认为是顺从于她的婆婆,她的婆婆是统治者。在西班牙裔家庭中,双方的祖父母可以和他们的孩子、孙子生活在同一屋檐下。母亲在家庭事务上通常会从祖母那里得到大量的支持,但这取决于每个家庭的独特动态。
			\subsection{}
			
			为了更好地理解人际关系是如何构建的,医疗服务提供者询问谁住在病人家里是极其重要的。谁是权威人士?在亚裔和西班牙裔的传统家庭中,父亲是主要的权威人物。他通常会对家庭以外的事情做出决定,在公共场合代表家庭发言,并签署同意书。通常是由女性掌管家庭生活。根据家族的情况,这位女族长可能是母亲,但也可能是母亲的母亲。因此医疗服务提供者需要询问母亲,“谁会给你关于抚养孩子的建议?”以及”谁将参与重要决策?”尤其是在亚裔和西班牙裔家庭,祖母经常决定使用传统药物和治疗方法,因此对病人的依从性有巨大的影响。
			\subsection{}
			\textbf{角色弹性和亲属关系}
			
			在与不同文化的家庭打交道时,卫生保健专业人员有必要了解角色灵活性和亲属关系的基本概念,以及这些概念如何影响家庭动态。美国的亲属关系结构是双向的,我们与父亲的家庭并不比母亲的家庭“更亲近”,反之亦然。在单一文化中,家族成员可以追溯到男性或女性祖先。例如,在中东,父系模式已经形成,因此家庭的归属通过父亲一方传递。有些美国印第安文化,比如纳瓦霍文化和霍皮特里布文化,是母系文化,通过母亲的家庭成员。在纳瓦霍部落,财产和特权由男性传给男性,但是母亲的兄弟会把财产和特权,传给自己姐妹的孩子。因此,纳瓦霍人的舅舅可能会把他的侄子带进医院,希望他们能够签署知情同意书。
			\subsection{}
			同样,在美国印第安人和非裔美国人家庭中,角色灵活性也是一个重要问题。当父母离开保留地去找工作时,印第安人祖父母抚养孙子女的情况并不少见。在非裔美国家庭中,母亲有时扮演父亲的角色,因此也是一家之主。此外,年龄较大的儿童有时充当年幼儿童的父母或看护者。非裔美国家庭的角色灵活性概念可以扩展到包括祖父、祖母、阿姨和堂兄弟姐妹所承担的父母角色。确定年龄较大的儿童是否会参与病人护理,并尽可能将他们纳入病人护理培训,这是一个好主意。对于所有多代家庭来说,这一点很重要。
			\subsection{}
			\textbf{家庭动力学和文化适应}
			
			最后,重要的是要考虑到家庭在文化适应过程中遇到的巨大压力,由于突然和激进的变化,在家庭动态。在一个新近迁移的家庭中,父母往往与起源国文化保持一致,而他们的子女可能更快地适应主流文化。这通常会导致两代人之间的冲突。例如,如果妻子开始外出工作,赚取收入和更大的独立性,父亲可能会失去作为一家之主的传统角色。同样,如果他的孩子很快接受新的主流文化的态度和价值观,他可能会发现更难与他们沟通。父母和祖父母可能会因为语言障碍而感到地位的丧失,特别是如果他们的孩子更快地学习主流文化的语言。这在医疗机构中尤其成问题,因为这些机构的责任转移到了年轻的家庭成员身上,而这些家庭成员能比他们的父母更好地驾驭医疗保健系统。如果儿童能够与卫生保健工作者用英语交流,他们可能会被要求为其父母翻译。这给家庭带来了一系列潜在的问题,包括羞耻感和背叛感,孩子们会向家庭以外的人传递个人信息。这就是为什么孩子不能当翻译的主要原因之一。
			\subsection{}
			\textbf{总结}
			
			因为文化适应和变化,对家庭动态的假设是有问题的; 今天在美国,来自各种文化的家庭呈现出各种各样的结构。可以说,现在已经没有所谓的“典型”家庭了。然而,我们可以预期,来自传统文化的家庭,没有被美国文化所同化,将倾向于重视家庭主义,显示出与欧美中产阶级家庭模式截然不同的家庭结构。基于文化的家庭动态有许多方面不在本文讨论的范围之内。了解跨文化家庭动态的一些最好的资源来自于心理健康和儿童发展领域。
						
		}
	\end{spacing}
	
\end{document}



 	 	