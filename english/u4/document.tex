\documentclass[]{article}
\usepackage[UTF8]{ctex}	% 中文语言包
\usepackage{setspace}
\usepackage{geometry}
\geometry{a4paper,left=2cm,right=2cm,top=1cm,bottom=1.5cm}

%opening
\title{英语课文翻译和解释-U4}
\author{周方全}

\begin{document}

\maketitle

\section{A}
\subsection{}
\begin{spacing}{2.0}
	{\Large At least six million obese teenagers in the United States are candidates for weight-loss surgery, experts estimate. Fewer than 1000 of them get it each year. Many of these adolescents already have complivations of obesity, like diabetes or high blood pressure. But doctors have been uncertain just how well surgery works for young patients, and whether they can handle the consequences, including a severely restricted diet.}\newline
\end{spacing}

专家估计,在美国,至少有600万肥胖青少年是减肥手术的候选对象。每年只有不到1000人能拿到。这些青少年中的许多人已经有肥胖的症状,比如糖尿病或高血压。但医生们一直不确定手术对年轻患者的效果如何,以及他们是否能处理好后果,包括严格限制饮食。


\subsection{}
\begin{spacing}{2.0}
	{\Large A new study provides some hopeful answers. Researchers followed 161 teenagers aged 13 to 19, and 396 adults aged 25 to 50, for five years after weight-loss surgery. The teenagers actually fared better than the adults. The adolescents lost at least as much weight, and were more likely to see high blood presure and diabetes ease or go away, the investigators reported on Wednesday in the New England Journal of Medicine." This really changes the game," said Dr. Amir Ghaferi, a bariatric aurgeon at the University of Michigan, who was not involved in the research. The paper, he said, added to evidence that obesity, like cancer, is best treated early, before long-term damage from related conditions, such as high blood pressure and diabetes, sets in.}\newline
\end{spacing}

一项新的研究提供了一些充满希望的答案。研究人员在减肥手术后对161名13岁至19岁的青少年和396名25岁至50岁的成年人进行了为期5年的跟踪调查。实际上,青少年的表现比成年人要好。研究人员周三在“新英格兰医学杂志”(New England Journal Of Medicine)上报道,这些青少年的体重至少减轻了同样多,而且更有可能看到高血压和糖尿病得到缓解或消失。密歇根大学(University Of Michigan)的减肥耳科医生阿米尔·加费里(Amir Ghaferi)博士说:“这真的改变了这个游戏。”他没有参与这项研究。他说,这篇论文补充了肥胖症和癌症一样,最好在高血压和糖尿病等相关疾病造成的长期损害开始之前及早治疗的证据。

\subsection{}
\begin{spacing}{2.0}
	{\Large To have the surgery,teenagers in the study had to meet the same criteria as adults: a body mass index of at least 35--for instance, a person who is 5 feet 2 inches tall and weights 192 pounds or more--and obesity-related health problems. Alternately, the adolescents could have a B.M.I. of at least 40--such as a person who is 5 feet 2 inches tall and weights at least 220 pounds--without other conditions linked to obesity.}\newline
\end{spacing}

要做手术,研究中的青少年必须符合与成年人相同的标准:身体质量指数至少为35-例如,身高5英尺2英寸(约合1.2厘米)、体重192磅或更多的人-以及与肥胖相关的健康问题。或者,这些青少年的BMI至少为40--比如身高5英尺2英寸、体重至少220磅的人--没有其他与肥胖有关的情况。

\subsection{}
\begin{spacing}{2.0}
	{\Large There is no exact data on the number of teenagers who meet those criteria in the United States, said Dr. Thomas Inge, chief of pediatric surgery at Children's Hospital Colorado and lead author of the new study. He estimated that about 8 percent of American teenagers would qualify. "These are not kids who are pleasantly plump," said Dr.John Morton, a bariatric surgeon at Yale University. "Once you have a B.M.I. of 30, it is really difficult to lose weight on your own."}\newline
	
\end{spacing}

科罗拉多州儿童医院(Children‘s Hospital Colorado)儿科主任、这项新研究的主要作者托马斯·英格(Thomas Inge)博士说,目前还没有关于美国符合这些标准的青少年人数的确切数据。他估计,大约8\%的美国青少年将符合条件。耶鲁大学(Yale University)的减肥外科医生约翰·莫顿(John Morton)博士说:“这些孩子不是令人愉快的丰满。”“一旦你的BMI达到30,你就真的很难靠自己减肥了。”

\subsection{}
\begin{spacing}{2.0}
	{\Large There is no other treatment that results in a substantial and sustained weightloss in people who are severely obese. But the operation, gastric bypass surgery, is demanding. Surgeons close off most of the stomach, leaving a small pouch, and reroute the intestines. Afterward, patients must eat tiny meals at frequent intervals for the rest of their lives.}\newline
\end{spacing}

对于严重肥胖的人来说,没有其他治疗方法可以使他们的体重持续大幅下降。但胃旁路手术要求很高。外科医生关闭了大部分胃,留下了一个小袋子,并改变了肠道的路线。之后,患者必须在他们的余生中频繁地吃少量的饭菜。

\subsection{}
\begin{spacing}{2.0}
	{\Large It is a scenario that gives many parents pause. Should their teenagers wait, hoping science will come up with a less drastic solution? Or should their children have the operation before even more serious harm to the body occurs? Adding to the quandary is the fact there are just six accredited bariatric surgery centers in pediatric hospitals. Although some adult programs operate on teenagers, most of such procedures are done in pediatric settings, Dr. Morton said. Adults, by contrast, may choose from 850 accredited medical centers for weight-loss surgery.}\newline
\end{spacing}

这是一个让很多家长犹豫的场景。他们的青少年是否应该等待,希望科学能拿出一个不那么激进的解决方案?还是他们的孩子应该在身体受到更严重的伤害之前做手术?雪上加霜的是,儿科医院只有六个经过认证的减肥手术中心。莫顿博士说,虽然有些成人项目是针对青少年的,但大多数此类手术都是在儿科环境中进行的。相比之下,成年人可以从850家经认证的减肥手术医疗中心中进行选择。

\subsection{}
\begin{spacing}{2.0}
	{\Large Although the new study included adolescents with B.M.I.s of 35, most were much heavier. The average was 50, the same as the average for adults in the study. For most of the adolescents, the surgery was a success. On average, they lost about a quarter of their body weight, enough to make life much easier and for most to shed health problems. The teenagers weighed 324 pounds on average when they had the surgery. Five years later, the figure was
	244 pounds. The adults weighed the same at the start and had a nearly identical result. An unlucky minority of patients -both adults and teenagers— did not fare as well. Some remained saddled	with high blood pressure or diabetes. A few lost almost no weight, or even gained weight, in the five years after the surgery.}\newline
\end{spacing}

虽然这项新的研究包括了BMI为35的青少年,但大多数都要重得多。平均为50人,与研究中成年人的平均水平相同。对于大多数青少年来说,手术是成功的。平均而言,他们减掉了大约四分之一的体重,足以让生活变得轻松得多,对大多数人来说,也足以减轻健康问题。这些青少年接受手术时的平均体重为324磅。五年后,这个数字是244英镑。成年人在开始时体重相同,结果几乎相同。少数不幸的患者-包括成年人和青少年-没有得到同样的结果。一些人仍然患有高血压或糖尿病。在手术后的五年里,一些人的体重几乎没有下降,甚至增加了。

\subsection{}
\begin{spacing}{2.0}
	{\Large The study did not randomly assign patients to have the surgery or not, which is the gold standard for clinical research. Since the adults had been obese for a much longer time, their condition might have been harder to treat with surgery. That makes it difficult to directly compare the teenagers and adults. But Dr. Ghaferi said the message is clear: It is best to intervene early. And if that is true, other experts wondered, what about even younger children?"What if an eight-year-old comes in, or a ten-year-old comes in, and they are severely obese? Why don't we offer this treatment and study the results?” Dr. Inge asked. Already he has operated on children who are severely obese and have related medical conditions. He said the question now is whether to operate even before a child develops high blood pressure, diabetes, or sleep apnea and it will require careful study, with researchers following children for years after surgery.}\newline
\end{spacing}

这项研究没有随机分配患者是否接受手术,这是临床研究的金标准。由于这些成年人肥胖的时间要长得多,他们的情况可能更难通过手术来治疗。这使得直接比较青少年和成年人变得困难。但加费里博士说,信息很明确:最好是及早干预。其他专家想知道,如果这是真的,那么更小的孩子呢?“如果一个8岁的孩子进来,或者一个10岁的孩子进来,他们严重肥胖怎么办?”为什么我们不提供这种治疗并研究结果呢?“英格博士问道。他已经为严重肥胖并有相关医疗条件的儿童做过手术。他说,现在的问题是,是否甚至在孩子患上高血压、糖尿病或睡眠呼吸暂停之前就进行手术,这需要仔细研究,研究人员在手术后对儿童进行多年的跟踪调查。

\subsection{}
\begin{spacing}{2.0}
	{\Large Even operating on teenagers raises issues that may not apply to adults. Can a miserable adolescent, for example, really give informed consent to such a drastic, life-changing operation? And can a teenager be expected to commit to following the very restricted diet required after the surgery, not
	to mention taking the needed v1tamins and minerals? "Are they prepared to do that for the rest of their lives?" asked David B. Sarwer, a psychologist at Temple University who works with bariatric surgery patients.}\newline
\end{spacing}

即使给青少年做手术也会引起成年人可能不会遇到的问题。例如,一个悲惨的青少年真的能在知情的情况下同意这样一个激进的、改变人生的手术吗?难道一个十几岁的孩子会在手术后坚持严格的饮食要求,更不用说服用必要的维生素1tamin和矿物质了吗?“他们准备好在余生这样做了吗?”坦普尔大学(Temple University)的心理学家大卫·B·萨维尔(David B.Sarwer)问道,他与减肥手术患者一起工作。

\subsection{}
\begin{spacing}{2.0}
	{\Large Dr. Sarwer added that substantial weight loss can have unexpected psychological consequences in teenagers. Severe obesity "sets adolescents up for stigmatization," and a severely obese teenager "is likely known by every other student in the high school not because she is a prom queen, but because she is physically the largest student in the school." Stigma often leaves teenagers isolated and lacking social skills, a deficit that can hinder their development even after surgery to lose weight.	And when formerly obese teenagers go to college, often they are so ashamed of having been fat that they keep it a secret.}\newline
\end{spacing}

萨威尔博士补充说,大幅减肥可能会给青少年带来意想不到的心理后果。严重肥胖“会让青少年蒙受耻辱”,而一个严重肥胖的青少年“很可能会被高中其他所有的学生所认识,不是因为她是舞会皇后,而是因为她身体上是学校里个子最大的学生。”“。耻辱往往会让青少年与世隔绝,缺乏社交技能,这一缺陷即使在减肥手术后也会阻碍他们的发展。当以前肥胖的青少年去上大学时,他们往往会为自己的肥胖而感到羞愧,所以他们对此保守了秘密。

\subsection{}
\begin{spacing}{2.0}
	{\Large Yet weight-loss surgery can be transformative. Eric	Decker, 33, a bartender	and freelance makeup artist in Detroit, had the operation in 2006 when he was 17. He was 5 feet 10 inches tall and weighed 385 pounds; no amount of dieting seemed to help. He tried to find a surgeon in South Carolina, where he lived, to operate on him, but no one would do it. He was referred to Dr. Inge, then at Cincinnati Children's Hospital.}\newline
\end{spacing}

然而,减肥手术可能会带来变革性的影响。33岁的埃里克·德克尔(Eric Decker)是底特律的一名调酒师,也是一名自由职业化妆师。2006年,17岁的他做了这个手术。他身高5英尺10英寸(约合1.78米),体重385磅(约合1.28公斤);再怎么节食似乎都无济于事。他试图在他居住的南卡罗来纳州找一位外科医生为他做手术,但没有人愿意做。他被转介给英格医生,当时他在辛辛那提儿童医院(Cincinnati Children‘s Hospital)就诊。

\subsection{}
\begin{spacing}{2.0}
	{\Large Mr. Decker lost more weight than most — he now weighs between 205 and 210 pounds. He speaks up now when someone speaks derisively of a person who is
	obese. He knows how it feels to be shunned for what meaical researcners now achronic disease, not a lifestyle choice. Without that experience, he said, "I don't think I would have that lens of compassion for people with their struggles."}\newline
\end{spacing}

德克尔比大多数人减掉了更多的体重--他现在的体重在205到210磅之间。现在,当有人嘲笑一个肥胖的人时,他就会大声疾呼。他知道被回避的感觉,因为现在的医疗研究是一种过时的疾病,而不是一种生活方式的选择。他说,如果没有这段经历,“我想我不会对那些正在挣扎的人抱有同情之心。

\section{B Culture and Its Influence on Nutrition and Oral Heaalth}
\subsection{}
\begin{spacing}{2.0}
	{\Large Food habits are one of the most complex aspects of human behavior, being determined by multiple motives and directed and controlled by multiple stimuli. Food acceptance is a complex reaction influenced by biochemical, physiological, psychological, social and educational factors. Metabolic conditions play an important role. Age, sex and mental state are factors of importance. People differ greatly in their sensory response to foods. The likes and dislikes of the individual with respect to food move in a framework of race, tradition, economic status and environmental conditions.}\newline
\end{spacing}

饮食习惯是人类行为中最复杂的方面之一,由多种动机决定,受多种刺激的指导和控制。食物接受是一种受生化、生理、心理、社会和教育等多种因素影响的复杂反应。新陈代谢条件起着重要作用。年龄、性别和精神状态是重要的因素。人们对食物的感官反应大不相同。个人对食物的好恶取决于种族、传统、经济地位和环境条件。

\subsection{}
\begin{spacing}{2.0}
	{\Large For most people food is cultural, not nutritional. A plant or animal may be considered edible in one society and inedible in another. Probably one of the most important things to remember in connection with the cultural factors involved in food habits is that there are many combination of food which will give same nutritional results.}\newline
\end{spacing}

对大多数人来说,食物是文化的,而不是营养的。一种植物或动物在一个社会中可能被认为是可食用的,而在另一个社会中可能被认为是不能食用的。关于饮食习惯中涉及的文化因素,可能最需要记住的一件事是,有许多组合的食物会产生同样的营养效果。

\subsection{}
\begin{spacing}{2.0}
	{\Large Culture consist of values, attitudes, habits and customs, acquired by learning which starts with the earliest experiences of childhood, much of which is not deliberately taught by anyone and which so thoroughly internalized that it is unconscious but 'goes deep'. Food habits are among the oldest and most deeply entrenched aspects of many cultures. Food and food habits as a basic part of
	culture serve as a focus of emotional association, a channel of love, discrimination and disapproval and usually have symbolic references. The sharing of food symbolizes a high degree of social intimacy and acceptance.}\newline
\end{spacing}

文化由价值观、态度、习惯和习俗组成,是通过从童年最早的经历开始学习而获得的,其中大部分不是任何人刻意传授的,而且是如此彻底地内化,以至于它是无意识的,但却“深入”。饮食习惯是许多文化中最古老、最根深蒂固的方面之一。食物和饮食习惯作为文化的基本组成部分,是情感联系的焦点,是爱、歧视和反对的渠道,通常具有象征意义。分享食物象征着高度的社会亲密感和接受度。

\subsection*{Food selection}
\subsection{}
\begin{spacing}{2.0}
	{\Large It has often been demonstrated that, in many areas of the world, people can live completely healthy lives despite the fact that, according to Western standards, their nutrition is inadequate. Authorities on the nutrition of people in Southeast Asia have pointed out that a diet which appears to be deficient is actually adequate, either because the people eat the most nutritious parts of plants
	and animals which elsewhere are thrown away as waste or because they have achieved an adaptation to the economical use of the food eaten. It seems unwise to use standards that are appropriate in industrialized societies as a measure of the nutritional adequacy of the diet of underdeveloped or primitive societies.}\newline
\end{spacing}

人们经常被证明,在世界上的许多地区,人们可以过上完全健康的生活,尽管按照西方标准,他们的营养不足。东南亚人民的营养主管部门指出,看似不足的饮食实际上是足够的,要么是因为人们吃的动植物中最有营养的部分在其他地方被当作垃圾丢弃了,要么是因为他们已经适应了所吃食物的经济使用。用工业化社会适用的标准来衡量不发达或原始社会的饮食营养充足似乎是不明智的。

\subsection{}
\begin{spacing}{2.0}
	{\Large The food consumed is determined by what is available. It is not surprising, therefore, to find considerable differences in food selection between rural and urban communities. Within both urban and rural communities, variations in food selection between families are also influenced by socio-economic status.}\newline
\end{spacing}

所消耗的食物是由可获得的食物决定的。因此,在农村和城市社区之间发现食物选择上的巨大差异也就不足为奇了。在城市和农村社区,家庭之间在食物选择上的差异也受到社会经济地位的影响。

\subsection{}
\begin{spacing}{2.0}
	{\Large Storage and distribution of food also matters. In the Middle East and Far East,	where the facilities for refrigeration, preservation, or storage are non-existent, any animal slaughtered must be consumed immediately, so the supply of first-class protein is irregular. In other regions, such as the Arctic and parts of Africa, meat is preserved by drying. In parts of Europe and the Middle East, fruits and vegetables are not preserved, so that they can be eaten only seasonally. In parts of Africa, poor storage methods have resulted in the development of toxic elements in rice.}\newline
\end{spacing}

食物的储存和分配也很重要。在中东和远东地区,没有冷藏、保存或储存的设施,任何被宰杀的动物都必须立即食用,因此一级蛋白质的供应是不正常的。在其他地区,如北极和非洲部分地区,肉类是通过干燥来保存的。在欧洲和中东的部分地区,水果和蔬菜没有保存,因此只能季节性食用。在非洲部分地区,糟糕的储存方法导致大米中有毒元素的产生。

\subsection{}
\begin{spacing}{2.0}
	{\Large Epidemiologists and public health workers who recognize a need for better nutrition, therefore, must take into account the traditional methods of growing and storing food. However, it is not enough to arrange to increase the available food supply. Changes will be acceptable only if they co-occur with the established food habits of the people.}\newline
\end{spacing}

因此,认识到需要更好营养的流行病学家和公共卫生工作者必须考虑到种植和储存粮食的传统方法。然而,仅仅安排增加可用的食物供应是不够的。只有在与人们既定的饮食习惯同时发生的情况下,改变才是可以接受的。

\subsection*{Dietary factors in oral health}
\subsection{}
\begin{spacing}{2.0}
	{\Large Numerous dental and dietary surveys have established a direct relationship between the prevalence of dental caries and the frequency with which fermentable carbohydrate in a sticky form is consumed. These have shown that alterations in the prevalence of caries have accompanied changes in the frequency with which sugar and sugar products have been consumed. Studies of populations in developing countries have shown that the prevalence of dental caries increases when the people	change from their traditional diet to the one that includes refined sugar and flour. Such foods appear to be universally acceptable not only for their pleasant taste but	also for their cheapness and the fact that they can be stored for comparatively long	time. So any study of the diet in relation to dental caries should not be restricted to an evaluation of its carbohydrate content. Other important factors, which vary in
	accordance with local custom or habit, should also be considered.}\newline
\end{spacing}

大量的牙科和饮食调查已经确定了龋齿患病率和粘性可发酵碳水化合物的消费频率之间的直接关系。这些研究表明,龋齿患病率的变化伴随着糖和糖类产品消费频率的变化。对发展中国家人口的研究表明,当人们从传统饮食转变为包括精制糖和面粉的饮食时,龋齿患病率会增加。这类食品似乎被普遍接受,不仅因为它们的味道宜人,而且还因为它们的价格便宜,而且它们可以储存相对较长的时间。因此,任何与龋齿有关的饮食研究都不应该局限于对其碳水化合物含量的评估。其他因当地风俗习惯而异的重要因素也应考虑在内。

\subsection*{Non- dietary factors in oral health}
\subsection{}
\begin{spacing}{2.0}
	{\Large These factors, including the selection and preparation of food, eating order and frequency, etc., determine the physical character of the food, which affects the vigor and duration of mastication. It in turn affects the rate of flow of saliva and the rate of clearance of food debris from the mouth.}\newline
\end{spacing}

这些因素,包括食物的选择和准备、进食顺序和频率等,决定了食物的物理性质,从而影响咀嚼的活力和持续时间。它反过来影响唾液的流动速度和食物残渣从口腔中的清除速度。

\subsection{}
\begin{spacing}{2.0}
	{\Large Many surveys conducted in underdeveloped countries claim that the physical nature of the food is the most significant factor in the initiation of dental caries. Further studies are required, however, to determine whether or not the incidence of caries can	be reduced by altering the physical character of the food in people who frequently consume sticky, refined sugar preparations.}\newline
\end{spacing}

在欠发达国家进行的许多调查声称,食物的物理性质是引发龋齿的最重要因素。然而,还需要进一步的研究来确定是否可以通过改变经常食用粘性精制糖制剂的人的食物的物理特性来降低龋齿的发生率。

\subsection{}
\begin{spacing}{2.0}
	{\Large In several underdeveloped countries the customary methods of cooking result in the incorporation of sand and ashes in the food. This results in extensive abrasion of the teeth. Occlusal surfaces are worn down below the maximum circumference of the teeth, and the proximal enamel breaks away, creating a space into which food becomes impacted. In this way heavy abrasion tends to decrease occlusal caries and predispose to proximal caries.}\newline
\end{spacing}

在一些不发达的国家,传统的烹饪方法导致在食物中加入沙子和灰烬。这会导致牙齿的广泛磨损。咬合面磨损到牙齿的最大周长以下,近端牙釉质脱落,形成一个食物受到影响的空间。在这种情况下,重度磨耗往往会减少咬合龋齿,并易患近端龋齿。

\subsection{}
\begin{spacing}{2.0}
	{\Large Fibrous or tough food will promote the clearance of food debris from the mouth, only if it is eaten at the end of a meal. In some
	countries this is a routine	practice. It is also widely advocated in most highly developed countries. In many underdeveloped countries and isolated communities, people have only one or two	meals a day. Between-meal food consumption is neither so frequent nor as ritualized as in many European countries. In most cases, the food requires vigorous mastiation, and the diet contains little or no refined carbohydrate. Under these circumstances the prevalence of caries is always very low.}\newline
\end{spacing}

纤维或坚硬的食物只有在用餐结束时食用,才能促进食物残渣从口腔中清除。在一些国家,这是一种常规做法。它在大多数高度发达的国家也得到了广泛的倡导。在许多不发达国家和与世隔绝的社区,人们一天只吃一到两顿饭。两餐之间的食物消费不像许多欧洲国家那样频繁,也不像许多欧洲国家那样仪式化。在大多数情况下,食物需要大力咀嚼,而饮食中只含有很少或根本没有精制碳水化合物。在这种情况下,龋齿的患病率总是很低的。

\end{document}

