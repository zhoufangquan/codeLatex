\documentclass[]{article}
\usepackage[UTF8]{ctex}	% 中文语言包
\usepackage{multicol}  % 用于实现在同一页中实现不同的分栏
\usepackage{amsmath}	% 公式功能包
\usepackage[shortlabels]{enumitem}	% 编号扩展功能包

% 伪代码的格式设置
%***************************************************************************************************
% 头文件部分
\makeatletter
\newif\if@restonecol
\makeatother
\let\algorithm\relax
\let\endalgorithm\relax
\usepackage[linesnumbered,ruled,vlined]{algorithm2e}%[ruled,vlined]{
\usepackage{algpseudocode}
\renewcommand{\algorithmicrequire}{\textbf{Input:}}  % Use Input in the format of Algorithm
\renewcommand{\algorithmicensure}{\textbf{Output:}} % Use Output in the format of Algorithm 
%***************************************************************************************************

\usepackage{titlesec}  % 自定义多级标题格式的宏包

% \titleformat{command}[shape]  % 定义标题类型和标题样式,字体
% {format}  % 定义标题格式:字号(大小),加粗,斜体 例如 \fontsize{20.75pt}\bfseries\centering
% {label}  % 定义标题的标签,即标题的标号等
% {sep}  % 定义标题和标号之间的水平距离
% {before-code}  % 定义标题前的内容

\titleformat{\section}[block]%定义标题类型和标题样式,字体
{\Large\bfseries}  % 定义标题格式:字号(大小),加粗(,斜体),居中
{\bfseries\arabic{section}}  % 定义标题的标签,即标题的标号等
{0.5em}  % 定义标题和标号之间的水平距离
{}  % 定义标题前的内容
[]  % 定义标题后的内容

\titleformat{\subsection}[block]  % 定义标题类型和标题样式,字体
{\large\bfseries}  % 定义标题格式:字号(大小),加粗,斜体
{\bfseries\arabic{section}.\bfseries\arabic{subsection}}  % 定义标题的标签,即标题的标号等
{0.5em}  % 定义标题和标号之间的水平距离
{}  % 定义标题前的内容
[]  % 定义标题后的内容

\titleformat{\subsubsection}[block]  % 定义标题类型和标题样式,字体
{\normalsize\bfseries}  % 定义标题格式:字号(大小),加粗,斜体
{\bfseries\arabic{section}.\bfseries\arabic{subsection}.\bfseries\arabic{subsubsection}}  % 定义标题的标签,即标题的标号等
{0.5em}  % 定义标题和标号之间的水平距离
{}  % 定义标题前的内容
[]  % 定义标题后的内容

\titleformat{\paragraph}[block]
{\small\bfseries}
{[\arabic{paragraph}]}
{1em}
{}
[] 

% \titleformat{\section}[block]{\LARGE\bfseries}{\Roman{section}}{1em}{Hello: }[]
% \titleformat{\subsection}[block]{\Large\itshape\mdseries}{\arabic{section}.\arabic{subsection}}{0.5em}{}[]
% \titleformat{\subsubsection}[block]{\normalsize\bfseries}{\arabic{subsection}-\arabic{subsubsection}}{0em}{}[]
% \titleformat{\paragraph}[block]{\small\bfseries}{[\arabic{paragraph}]}{1em}{}[]



\usepackage{setspace}  % 设置行间距
\usepackage{geometry}  % 设置一些页面格式,还没开发安全
\geometry{a4paper,left=3.0cm,right=3.0cm,top=2.25cm,bottom=2.0cm}

% 正文字体的设置
\PassOptionsToPackage{no-math}{fontspec}
\usepackage{mathspec}
\setmainfont{Times New Roman}  % 正文英文字体的设置
\setCJKmainfont{SimSun}[AutoFakeBold,ItalicFont=KaiTi]  % 正文中文字体的设置
\setCJKsansfont{SimHei}%对应sf无衬线
\setCJKmonofont{FangSong}%对应tt打字机
%\newCJKfontfamily{\kaishu}[AutoFakeBold={2.17}]{STXingkai}


% 摘要格式的设置
\usepackage{tikz}
\usetikzlibrary{shapes,shadows}
\tikzstyle{abstractbox} = [draw=black, fill=white, rectangle, 
inner sep=20pt, style=rounded corners, drop shadow={fill=black,
	opacity=0.5}]
\tikzstyle{abstracttitle} =[fill=white]

\newcommand{\boxabstract}[2][fill=white]{
	\begin{center}
		\begin{tikzpicture}
			\node [abstractbox, #1] (box)
			{\begin{minipage}{0.88\linewidth}
					\setlength{\parindent}{2mm}
					\small #2
			\end{minipage}};
			\node[abstracttitle, right=10pt] at (box.north west) {\textbf{摘要}};
		\end{tikzpicture}
	\end{center}
}
	
	
%\newsavebox{\myabstractbox}
%\providecommand{\abstractnode}[2]{
	%	\begin{tikzpicture}%
		%		\node [abstractbox, fill=#1](box)%
		%		{#2};%
		%		\node[abstracttitle, right=10pt] at (box.north west) {Abstract};
		%	\end{tikzpicture}
	%}
%
%
%\newenvironment{abstractbox][1][white]{
%		\begin{center}%
	%			\def\abs@bgcok{#1}%
	%			\begin{Irbox}{\myabstractbox}
		%				\begin{minipage}{.80\linewidth}%% lparindent2em%
			%					\footnotesize #2
			%				\end{minipage}
		%			\end{Irbox}
	%%			\abstractnode{\abs@bgcol}{\usebox{\myabstractbox}}%
	%		\end{center}%
%	}
		
		
		
		
		
		


%opening
\title{\textbf{波普尔证伪主义科学发展模式研究}}
\author{{\small 计算机工程与技术学院 \quad\quad 周方全 | 202121081229 \quad\quad \textbf{邮件地址}:2542154447@qq.com}}
\date{}


\begin{document}
\maketitle  % 生成标题

\boxabstract{	
	{ 波普尔证伪主义认为,科学理论必须是可证伪的,科学知识总是经得起试探和猜想的。自1960年以来,波普尔的著作包含了这些观点的一些本质发展,并朝着认识论乐观主义迈进了一步。尽管我们不能证明科学理论是正确的,但科学的目的是寻求真理,我们没有理由对更接近真理的概念持怀疑态度。我们的知识可以增长,科学也可以进步。本文通过论述波普尔的证伪主义的主要内容,阐述波普尔的证伪主义的合理性与优点,并根据波普尔证伪主义中存在的问题提出了一些批评与反驳。\rm} \newline \newline	
	{   \textbf{关键词}:\ 波普尔; 科学; 证伪主义; 逻辑 \rm}
}

\begin{multicols}{2} 
	
\section{简介}
近代西方经验论认为,人们的感觉经验是可靠的。进而人们可以在大量经验事实的基础上进行分析,最终得出一般的、抽象的结论,从具体的、特殊的事例中总结出概念与普遍规律,这样的过程也就是归纳的过程,而归纳法一直被看作一种合理的获取知识的方式。然而休谟的质疑则明确指出人们难以通过归纳来获得普遍有效的科学知识,而波普尔提出证伪、试错的方法来代替归纳法\cite{王飞2015波普尔证伪主义科学发展模式的解读}。

卡尔·波普尔(1902-1994)是最伟大的科学哲学家之一,他把证伪主义说成是科学与非科学的分界线。他就读于维也纳大学,在那里他接触到了弗洛伊德和阿德勒提出的精神分析理论以及马克思的理论。在维也纳听爱因斯坦关于相对论的演讲时,他对爱因斯坦理论中的“批判精神”印象深刻。而根据波普尔的说法,这种精神在马克思和弗洛伊德的理论中完全不存在,使这些理论不受反驳,这具有至关重要的意义。

一般在我们的常识中,科学的东西是正确的,而错误的则是不科学的。但按照波普尔的观点,科学不再是正确与错误的分界标准,例如,像数学这样通过严密的逻辑推理得出的结果是不会出现错误的,这必然包含着真理,但是科学是包含着经验事实的,而数学因不包含经验内容所以被划入了非科学的领域。非科学包含着真理,科学同样可错,波普尔对科学与非科学的划分提出自了己独到的见解。而另一方面,科学与伪科学之间也同样被重新规定了,科学因其可被证伪所以不再是绝对正确的,而不可错则恰恰是那些为科学如迷信等的特征。

波普尔如此的分界标准,把可证伪性变为了科学的标志,而那些绝对无误的东西则是伪科学。所以,面对科学,人们的态度不应该是崇拜,“科学的精神不是昭示无法反驳的真理,而是在坚持不懈的批判过程中寻找真理”。这也恰恰体现出科学的精神——批判,批判与信仰的对立也就是科学与伪科学之对立的根本所在。“科学只有在寻求反驳中才有希望学到东西和获得进步。”\cite{波珀1986科学发现的逻辑}

\section{波普尔证伪主义的特点}

波普尔证伪主义认为,我们永远不会得到真理,只是一直接近真理的路上。我们现在所认为的真理只不过是一个又一个的假说。只是随着科学的发展,我们的假说在不断地接近真理。

波普尔认为,真理必须从相信真理的主观经验中分离出来,真理和确定性的概念不能混淆。科学的目的不能是寻求确定性,因为所有的知识都是错误的和不确定的,但对真理的追求仍然存在。就像古希腊哲学家色诺芬尼(公元前570-475年)\cite{2014Falsifications}指出的那样,波普尔支持的客观真理理论使我们认为,我们寻找真理,但可能永远达不到真理,或者当我们达到真理时我们也不能辨认出这就是真理。波普尔使用了一个著名的比喻,将真理的状态比作被云包裹的山顶。登山者可能不仅很难登顶,即便当他登顶了,他可能会认为:在云中,他可能无法区分主峰和周围的小峰。倘若,当他还没有登顶的时候,他可能会认为:当他看到一个更高的地方,他可以因此决定继续朝着那个方向前进。一般来说,我们没有真理的标准,也就是辨别真理的程序,但我们确实有走向真理的标准。

然而,如果只局限于现实的琐碎方面,那么对真理的追求也可能成为次要的理想,因为在科学中,我们寻求的不仅仅是简单的真理。就像在数学中,我们不满足于说二加二等于四一样,在科学中,我们也渴望有趣且难以获得的真理。因此,我们更喜欢一个大胆的猜测,即使它最终被证明是错误的,而不是一系列正确但乏味的断言。从失败中,我们可以学到更多关于真理的东西;我们可以消除错误,更接近真理。波普尔补充说,“这是新的元素为了接近真理,以纠正以前理论的错误,但通常是不够的。”当然,需要一种新的理论来解决先前理论的困难,但这一理论也必须使预测以前从未观察到的事实成为可能,并通过关于这些新预测的一些测试。一系列不间断的被驳斥的理论很快就会让我们困惑和无助。我们需要成功和经验的佐证,以便了解我们是否走在正确的道路上,也理解成功驳斥的意义。

波普尔证伪主义在不断证伪中推翻一个个假说以更加的接近真理——达到另一个更接近真理的假说。在反驳逻辑实证主义的归纳逻辑中波普尔曾提到,我们的理论内容越是丰富,这个理论为真的概率就越低(因为偶然真理为真是具有概率的,假设a理论为真的概率为0.9,b理论为真的概率为0.8,那么为ab结合而产生的c理论为真的概率变为0.72,显然c理论更丰富,但其为真的概率更低),而在证伪主义中,我们也看到相似的地方,因为每一句话都有可能被证伪,那一个理论的内容越多,它就有更多的机会被证伪,而且理论内容越多,所提供的经验事实也就越多,那么可证伪性就和理论的丰富性成正相关的关系,可证伪性也就和该理论为真的可能性呈现反比的关系。证伪主义的价值不仅仅体现在它能使人们正确看待现有的科学理论其实是一种假说,这样人们就用一个又一个的假说来推翻前一个从而获取新的知识,还在于它能预测新的知识。


\section{波普尔证伪主义存在的问题}

波普尔证伪主义的基本出发点:科学进步是通过一系列错误(或假设)的理论取得的,这些理论通过纠正逐渐被证伪或者支持新的结果或经过验证的预测来实现越来越接近真理。为了准确地解释他的意思,我们提出两个理论,A和B。这两个理论都是错误的(A可以被认为是更早的理论,B是后来取代它的理论)。并指出如果在从A到B的过程中,若错误结果集被减少而不损害真结果集或者真结果集被加强而不同时增加假结果集,则B比A更接近真理。这个定义在逻辑上似乎很合适,即在提供相同数量的真实信息的情况下,一种错误理论比另一种理论包含的错误更少,或者在提供相同的错误信息的情况下,包含更多的真实信息。不幸的是,几年后,一些批评家\cite{1974On,2009Out}表明,波普尔为接近真理而建立的条件没有一个是可以验证的,因为理论的真实结果和错误结果一起增加和减少。

然而,两种错误的科学理论无法进行比较\cite{2014Falsifications}。纵观科学史,似乎有理由认为,逐渐被证伪的理论仍然可以被认为是对未知真理越来越好的近似。哥白尼的天文系统被认为比托勒密的天文系统更好,牛顿和爱因斯坦的理论被认为更好。我们也可以举出一些微不足道的错误陈述的例子,比如:“太阳系中有10颗行星”这一说法似乎比“太阳系中有10,000颗行星”这一说法的错误更少,因此更接近真理。关于这点,波普尔承认他的逻辑错误,并试图纠正逼真的最初定义,他的一些学生和其他学者在接下来的几年里也正在致力于这方面的研究。

另外,波普尔认为证伪主义的试错法优于归纳法,但是经过分析我们可以发现,并非如此。波普尔强调的是经验事实可以对全称命题进行证伪,但从另一方面,无论多少事实也不能对单称命题进行证伪,而一个事实就可以对单称命题进行证实。从这一点来看,归纳法与证伪主义的试错法都有对方做不到的地方,不应该简单地认为试错法比归纳法更优越。此外,科学上使用的归纳法并不同于简单枚举法,人们不会通过只找到一些可以支持这个理论的论据就说该理论被证明了,而倡导归纳法的人也会注意归纳法的局限性而去搜寻反例去不断完善理论。可见,归纳法与证伪主义在一些方面的思想内涵是一致的。其实在科学中,证伪与证实,归纳与反驳,本就是不可分的,在科学研究与学习中,我们本就是通过归纳得出最初的结论,不仅看到证实理论的一面,也要看到可以证伪理论的一面,才不断丰富理论,而强行的认为证伪优于证实未免有些牵强。
\section{总结}

证伪主义的确有值得肯定的一方面,但是科学不仅仅是证伪,我们也需要肯定的判断与知识,即证实。在波普尔看来,证实只是证伪过程的步骤之一,但是科学不能总是充满否定性的答案,否则面对这个世界,我们的内心只能是一片茫然与疑问。波普尔的证伪主义针对逻辑实证主义与归纳法进行了批判与论述,科学的精神、理性的精髓就是在于不断地证伪和批判,理性的标准也在应该在于能否接受批判、运用批判。他的证伪主义给科学提出了分界的新标准,其批判理性主义更是不同于传统的理性主义与非理性主义,融入了非理性的因素。此外,正是因为科学理论只是假说而不是真理,我们永远不能绝对地确定我们是否已经发现了它。我们的解答也只是暂时性的解答,而科学能进步也就在于我们无法掌握不到真理而不断逼近真理,同样的科学哲学观,也影响到了波普尔的历史哲学等――没有人掌握真理,也就没有人是权威,这样一种不屈服于任何人、任何事情的态度才是科学的真正精神所在。

在波普尔的眼里,我们都不能得真理我们得到的都是假说。大家都半斤八两,大家也可以在科学的王国里畅所欲言。这种自由的科学国度还是挺有意思和令人向往的。
% 添加参考文献
\bibliographystyle{IEEEtran}    % 设置文献样式
\bibliography{paper.bib}    % 设置文献的文件路径

\end{multicols}
\end{document}
	
